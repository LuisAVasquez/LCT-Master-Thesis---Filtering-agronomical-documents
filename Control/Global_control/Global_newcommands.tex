\newcommand{\WARNING}[1]{
\colorbox{pink}{#1}
 }
\newcommand{\TODO}[0]{\WARNING{TODO}}
\newcommand{\todo}[1]{\WARNING{TODO: #1}}

\colorlet{body}{black!80!white}
\newcommand{\cvtag}[1]{%
  \tikz[baseline]\node[anchor=base,draw=body!30,rounded corners,inner xsep=1ex,inner ysep =0.75ex,text height=1.5ex,text depth=.25ex]{#1};
}

\newcommand{\putInBox}[1]{
\begin{tcolorbox}[colback=mylightblue,colframe=gray!50!black] 
#1
\end{tcolorbox}

}


\definecolor{mylightblue}{RGB}{207,226,243}
\definecolor{mylightyellow}{RGB}{252,229,205}


\newcommand{\appendixName}[0]{Appendix{}}

\newcommand{\INRAE}[0]{INRAE}
\newcommand{\MAIAGE}[0]{MaiAGE}
\newcommand{\PESV}[0]{PESV}
\newcommand{\VSI}[0]{VSI}
\newcommand{\bibliome}[0]{Bibliome}
\newcommand{\textclassification}[0]{Text Classification}
\newcommand{\tokenization}[0]{Tokenization}
\newcommand{\wordpiece}[0]{Word Piece}
\newcommand{\bpe}[0]{Byte-Pair Encoding}
\newcommand{\finetuning}[0]{Fine-tuning}
\newcommand{\PET}[0]{PET}



\newcommand{\fOne}[0]{\ensuremath{F_1}}
\newcommand{\fTwo}[0]{\ensuremath{F_2}}
\newcommand{\fBeta}[0]{\ensuremath{F_\beta}}
\newcommand{\auc}[0]{AUC}

\newcommand{\contentType}[0]{Content Source}

\newcommand{\trafilaturaTitle}[0]{Title}
\newcommand{\trafilaturaAbstract}[0]{Abstract}
\newcommand{\trafilaturaFulltext}[0]{Full text}
\newcommand{\translationTitle}[0]{Translated Title}
\newcommand{\trafilatura}[0]{Trafilatura}

\newcommand{\keyphrases}[0]{Phrases with Keywords}

\newcommand{\keyphrasesAbstractOnly}[0]{Phrases with Keywords (Abstract)}
\newcommand{\keyphrasesAbstractOC}[0]{Phrases with Keywords + O.C (Abstract)}

\newcommand{\keyphrasesFulltextOnly}[0]{Phrases with Keywords (Full text)}
\newcommand{\keyphrasesFulltextOC}[0]{Phrases with Keywords + O.C (Full text)}


\newcommand{\sanbaTitle}[0]{Title}
\newcommand{\sanbaAbstract}[0]{Abstract}

\newcommand{\sanba}[0]{Sanba}
\newcommand{\BERT}[0]{BERT}

\newcommand{\balanced}[0]{Balanced \finetuning{}}
\newcommand{\unbalanced}[0]{Unbalanced \finetuning{}}
\newcommand{\petFifty}[0]{PET 50}
\newcommand{\petOneHundred}[0]{PET 100}
\newcommand{\petTwoHundred}[0]{PET 200}
\newcommand{\petFiveHundred}[0]{PET 500}
\newcommand{\petThousand}[0]{PET 1000}


\newcommand{\bertbase}[0]{BERT-base}
\newcommand{\bertmultilingual}[0]{mBERT}
\newcommand{\bertbiolinkbert}[0]{Bio-Link-BERT}
\newcommand{\bertscibert}[0]{SciBERT}
\newcommand{\bertroberta}[0]{RoBERTa-base}
\newcommand{\bertxlmroberta}[0]{XLM-RoBERTa}

\newcommand{\neuralNetwork}[0]{Neural Network}

\newcommand{\mycoloredcell}[1]{
\cellcolor[rgb]{\fpeval{1-#1}, \fpeval{#1}, 0} \fpeval{round(100*#1, 2)} \%
}

\newcommand{\toPercentage}[1]{\fpeval{round(100*#1, 2)} \%}


%%% Colored table

\newcommand{\coloredTable}[1]{

\centering
\resizebox{\paperwidth}{!}{
\csvreader[
    tabular=|p{4cm}|*{7}{c|}p{4cm}|,
    table head=\hline %\rowcolor{gray!20} 
    \contentType{} & \unbalanced{} & \balanced{} & \petFifty{} & \petOneHundred & \petTwoHundred{} & \petFiveHundred{} & \petThousand{} & \contentType{}  \\ \hline,
    %late after line=\\ \hline
    late after line=\\, late after last line=\\\hline
    ]%
    {#1}{}%
    {
    \csvcolii
    & \cellcolor[rgb]{\fpeval{(1-\csvcoliii)}, \fpeval{\csvcoliii}, 0} \fpeval{round(100*\csvcoliii, 2)}\% 
    & \cellcolor[rgb]{\fpeval{1-\csvcoliv}, \fpeval{\csvcoliv}, 0} \fpeval{round(100*\csvcoliv, 2)}\% 
    & \cellcolor[rgb]{\fpeval{1-\csvcolv}, \fpeval{\csvcolv}, 0} \fpeval{round(100*\csvcolv, 2)}\%
    & \cellcolor[rgb]{\fpeval{1-\csvcolvi}, \fpeval{\csvcolvi}, 0} \fpeval{round(100*\csvcolvi, 2)}\%
    & \cellcolor[rgb]{\fpeval{1-\csvcolvii}, \fpeval{\csvcolvii}, 0} \fpeval{round(100*\csvcolvii, 2)}\% 
    & \cellcolor[rgb]{\fpeval{1-\csvcolviii}, \fpeval{\csvcolviii}, 0} \fpeval{round(100*\csvcolviii, 2)}\%
    & \cellcolor[rgb]{\fpeval{1-\csvcolix}, \fpeval{\csvcolix}, 0} \fpeval{round(100*\csvcolix{}, 2)}\%
    & \cellcolor{white} \csvcolii
    }
}


}


%%% Split Statistics table

\newcommand{\splitsFinetuningTable}[1]{

    \centering
    \resizebox{\textwidth}{!}{
    \csvreader[
        tabular=|p{4cm}|*{9}{c|},
        table head=\hline
        \contentType{}
        & Train-Size
        & Positives
        & Negatives
        & Dev-Size
        & Positives
        & Negatives
        & Test-Size
        & Positives
        & Negatives
        \\ \hline,
        late after line=\\, 
        late after last line=\\\hline
        ]%
        {#1}{}%
        {
        \csvcolii
        & \csvcoliii
        & \csvcoliv
        & \csvcolv
        & \csvcolvi
        & \csvcolvii
        & \csvcolviii
        & \csvcolix
        & \csvcolx
        & \csvcolxi
        }
    }
}

\newcommand{\splitsPetTable}[1]{

    \centering
    \resizebox{\textwidth}{!}{
    \csvreader[
        tabular=|p{4cm}|*{10}{c|},
        table head=\hline
        \contentType{}
        & Train-Size
        & Positives
        & Negatives
        & Dev-Size
        & Positives
        & Negatives
        & Test-Size
        & Positives
        & Negatives
        & Unlabeled
        \\ \hline,
        late after line=\\, 
        late after last line=\\\hline
        ]%
        {#1}{}%
        {
        \csvcolii
        & \csvcoliii
        & \csvcoliv
        & \csvcolv
        & \csvcolvi
        & \csvcolvii
        & \csvcolviii
        & \csvcolix
        & \csvcolx
        & \csvcolxi
        & \csvcolxii
        }
    }
}




%%% Epchs Statistics table

\newcommand{\epochTable}[1]{

    \centering
    \resizebox{\textwidth}{!}{
    \csvreader[
        tabular=|p{4cm}|*{8}{c|},
        table head=\hline
        Model 
        & \trafilaturaTitle{}
        & \trafilaturaAbstract{}
        & \trafilaturaFulltext{}
        & \translationTitle{}
        & \keyphrasesAbstractOnly{}
        & \keyphrasesAbstractOC{}
        & \keyphrasesFulltextOnly{}
        & \keyphrasesFulltextOC{}
        \\ \hline,
        late after line=\\, 
        late after last line=\\\hline
        ]%
        {#1}{}%
        {
        \csvcoli
        & \csvcolii
        & \csvcoliii
        & \csvcoliv
        & \csvcolv
        & \csvcolvi
        & \csvcolvii
        & \csvcolviii
        & \csvcolix
        }
    }
}
