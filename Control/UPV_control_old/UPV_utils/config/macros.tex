
%% KODIFIKAZIOA %%
% \usepackage[utf8x]{inputenc}
\usepackage[T1]{fontenc}

%% FONTS %%
\usepackage{libertine}
\usepackage{inconsolata}

%% HYPERREFERENCES %%

\usepackage[
hyperindex,
bookmarks,
colorlinks=true,
citecolor=blue,
urlcolor=blue,
linkcolor=blue,
hypertexnames=true,
pagebackref %now added when importing biblatex
]{hyperref}



\renewcommand*{\backref}[1]{}
\renewcommand\backreftwosep{, }
\renewcommand\backrefsep{, }

%% MISC. %%
\usepackage{appendix}
\usepackage{placeins}
\usepackage[figuresright]{rotating}
\usepackage{graphicx}
\usepackage{subfig}
\usepackage{float}
%\floatstyle{boxed}
\restylefloat{figure}
\usepackage{xcolor, graphicx}
\usepackage{multirow}
\usepackage{pdfpages}
\usepackage{xspace}
\usepackage{microtype}
\usepackage{longtable}
\setsecnumdepth{subsubsection}
\maxtocdepth {subsection}
\setlength{\parskip}{5pt}
\usepackage{enumerate}
\makeatletter
\renewcommand{\counterwithin}{\@ifstar{\@csinstar}{\@csin}}
\makeatother
\usepackage{doi}
\usepackage{amsmath, amssymb, mathrsfs, mathtools}
\usepackage[capitalise]{cleveref}



\ifdefined\euskaraz
\newcommand{\upvehu}{Euskal Herriko Unibertsitatea UPV/EHU}
\newcommand{\gapizenburua}{Gradu Amaierako Lana}
\newcommand{\malizenburua}{Master Tesia}
\newcommand{\thesisizenburua}{Doktorego Tesia}
\newcommand{\informatikafakultatea}{Informatika Fakultatea}
\renewcommand{\abstract}{Laburpena}
\newcommand{\zuzendariaktestua}{Zuzendariak}
\fi
\ifdefined\castellano
\newcommand{\upvehu}{Universidad del País Vasco UPV/EHU}
\newcommand{\gapizenburua}{Trabajo de Fin de Grado}
\newcommand{\malizenburua}{Tesis de Máster}
\newcommand{\thesisizenburua}{Tesis Doctoral}
\newcommand{\informatikafakultatea}{Facultad de Informática}
\renewcommand{\abstract}{Resumen}
\newcommand{\zuzendariaktestua}{Dirección}
\fi
\ifdefined\english
\newcommand{\upvehu}{University of the Basque Country UPV/EHU}
\newcommand{\gapizenburua}{Bachelor Thesis}
\newcommand{\malizenburua}{Master Thesis}
\newcommand{\thesisizenburua}{PhD Dissertation}
\newcommand{\informatikafakultatea}{Informatics Faculty}
\renewcommand{\abstract}{Abstract}
\newcommand{\zuzendariaktestua}{Advisors}
\fi

\usepackage[font=small,labelfont=bf]{caption}

\ifdefined\euskaraz
\usepackage[basque]{babel}
% \hyphenation{Ko-man-do-in-ter-pre-ta-tzailea ba-te-ra-ga-rri-ta-suna ezau-garri}

\addto\captionsbasque{
	\renewcommand{\contentsname}{Gaien aurkibidea}
	\renewcommand{\listfigurename}{Irudien aurkibidea}
	\renewcommand{\listtablename}{Taulen aurkibidea}
	\renewcommand{\appendixname}{Eranskina}%
	\renewcommand{\appendixpagename}{Eranskinak}
	\renewcommand{\appendixtocname}{Eranskinak}
	\renewcommand{\bibname}{Bibliografia}
	\renewcommand{\tablename}{Taula}
	\renewcommand{\figurename}{Irudia}
}

\renewcommand*{\backrefalt}[4]{%
	\ifcase #1%
	\or Ikusi #2 orrialdea.%
	\else Ikusi #2 orrialdeak.%
	\fi%
}

%% Captionak euskarazko ordenean
\DeclareCaptionLabelFormat{euskaraz}{#2\bothIfSecond{\nobreakspace}{#1}}
\captionsetup{labelformat=euskaraz}
\fi

\ifdefined\castellano
\usepackage[spanish]{babel}
\addto\captionsspanish{
	\renewcommand{\contentsname}{Índice de contenidos}
	\renewcommand{\listfigurename}{Índice de figuras}
	\renewcommand{\listtablename}{Índice de tablas}
	\renewcommand{\appendixname}{Apéndice}%
	\renewcommand{\appendixpagename}{Apéndices}
	\renewcommand{\appendixtocname}{Apéndices}
	\renewcommand{\bibname}{Bibliografía}
	\renewcommand{\tablename}{Tabla}
	\renewcommand{\figurename}{Figura}
}
\renewcommand*{\backrefalt}[4]{%
\ifcase #1%
\or Ver página #2.%
\else Ver páginas #2.%
\fi%
}
\fi

\ifdefined\english
\usepackage[english]{babel}
\renewcommand*{\backrefalt}[4]{%
	\ifcase #1%
	\or See page #2.%
	\else See pages #2.%
	\fi%
}
\fi

\let\theoldbibliography\thebibliography
\renewcommand\thebibliography[1]{
	\theoldbibliography{#1}
	\setlength{\parskip}{0pt}
	\setlength{\itemsep}{4pt plus 0.3ex}
	\small
}

\iffalse
\fi