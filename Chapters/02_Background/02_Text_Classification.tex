\customHeader{1}{\textclassification{}}
\label{02_text_classification}

\textclassification{} is a fundamental task in \gls{nlp} that involves automatically assigning predefined categories or labels to text documents based on their content. The goal of \textclassification{} is to develop computational models that can accurately classify text documents into predefined categories, enabling automated organization, retrieval, and analysis of large volumes of textual data.

Formally, given a set $V$ called the \emph{Vocabulary}, and a set $C$ of \emph{categories}, a \emph{\textclassification{} System} is a function 
$$f: V^*\times C \mapsto \{ \texttt{True}, \texttt{False} \}$$

where $V^*$ is the set of finite sequences of elements of $V$, that is, the set of \emph{documents}. 
While the general definition of \textclassification{} accommodates multi-label classification scenarios, the specific focus of this project is on \emph{Binary \textclassification{}}. In this particular case, the classification task involves assigning documents to one of two mutually exclusive categories 
\myparencite{Shen2009_text_classification_formal_definition,text_classification_embeddings_survey}.

In our particular scenario, we will be handling entire documents for analysis rather than, for example, individual words or proper names. Thus, the task is termed \emph{Document Classification}.
In the context of Document Classification, the fundamental objective is to leverage a training set, comprising a collection of training documents $(d_i, c_i)$, and employ a learning method or algorithm to derive a \emph{Classifier} $\gamma$. The primary purpose of this classifier is to establish a mapping from documents to two classes, denoted as 
$$\gamma : V^* \mapsto \{1, 0\}$$
