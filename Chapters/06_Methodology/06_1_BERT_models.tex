\customHeader{1}{BERT Models}
\label{06_bert_models}



As previously mentioned in \headerName{} \ref{02_bert_models_for_text_classification}, the \gls{bert} architecture from \mytextcite{BERT_paper} has been customized for various tasks, resulting in a wide array of models to select from. For our project, we have opted to employ six different \gls{bert} models, based on their specialized task and the characteristics of our data.
%Since our data is multilingual in nature, 
We utilize \gls{bert} models that differentiate between upper case and lower case characters (\texttt{cased} versions). Additionally, due to computational constraints, we opt for the basic (\texttt{base}) versions of the models rather than the \texttt{large} ones (Table \ref{tab:06_bert_sizes}).

\begin{table}
    \centering
    \resizebox{\textwidth}{!}{
\begin{tabular}{|l|c|c|c|c|c|c|}
\hline
\textbf{Model} & \textbf{Languages} & \textbf{Tokenization} & \textbf{Encoder Layers} & \textbf{Attention Heads} & \textbf{Vocabulary} & \textbf{Parameters} \\
\hline
\bertbase{} & 1 & WordPiece & 12 & 12 & 30k & 110M \\
%BERTLarge & 1 & WordPiece & 24 & 16 & 30k & 335M \\
\bertmultilingual{} & 104 & WordPiece & 12 & 12 & 110k & 172M \\
\bertbiolinkbert{}& 1 & WordPiece & 12 & 12 & 30k & 110M \\
\bertscibert{}& 1 & WordPiece & 12 & 12 & 30k & 110M \\
\bertroberta{} & 1 & BPE & 12 & 8 & 50k & 125M \\
%RoBERTa & 1 & bBPE & 24 & 16 & 50k & 355M \\
\bertxlmroberta{} & 100 & SentencePiece & 12 & 12 & 250k & 270M \\
%XLM-R & 100 & SPM & 24 & 16 & 250k & 550M \\
\hline
\end{tabular}
}
    \caption{Sizes of \BERT{} models used for this project. Adapted from \mytextcite{xlm_roberta}}
    \label{tab:06_bert_sizes}
\end{table}


\customHeader{2}{\bertbase{}}
\label{06_bert_base}
The original \gls{bert} model developed by Google in \mytextcite{BERT_paper}. It was trained on a large corpus of English text from Wikipedia and Google’s BooksCorpus dataset, using a \gls{wp} tokenizer.

\customHeader{2}{\bertmultilingual{}}
\label{06_bert_multilingual}
The Multilingual \gls{bert} \myparencite{BERT_paper} is a version of the original \gls{bert} model that is designed to handle text in multiple languages. It was trained on the top 104 languages with the biggest Wikipedia sizes by number of articles, using a \gls{wp} tokenizer. 

\customHeader{2}{\bertbiolinkbert{}}
\label{06_bert_biolinkbert}

This model was introduced in \mytextcite{linkbert_biolinkbert} as the result of training a classical \gls{bert} architecture while enriching the context of the documents by using two corpora of documents linked to one another by hyperlinks. For documents $doc_1, doc_2$ linked by a hyperlink $doc_1 \to doc_2$, during the training phase, the sections of $doc_2$ that are reachable by the hyperlink will be concatenated to their corresponding sections from $doc_1$ for the Next Sentence Prediction task (See \headerName{} \ref{02_next_sentence_prediction}). This process increases the context available for the model to learn embeddings. 

The authors present a model for the general domain (Link-BERT), trained on the English Wikipedia with hyperlinks; and a model for the biomedical domain (\bertbiolinkbert{}), trained on PubMed\footnote{\url{https://pubmed.ncbi.nlm.nih.gov/}} abstracts (in English) containing hyperlinks, and used a \gls{wp} tokenizer. As of the culmination of our work, \bertbiolinkbert{} is still one of the top performing \gls{bert} models in the \gls{blurb} \myparencite{blurb_biomedical_ranking}.


\customHeader{2}{\bertscibert{}}
\label{06_bert_scibert}
This model was introduced in \mytextcite{scibert} as a specialized model for scientific texts. It was trained with the classical \gls{bert} architecture using 1.14 million papers in English from Semantic Scholar\footnote{\url{https://www.semanticscholar.org/}}, $82\%$ of which belong to the biomedical domain. The authors use the full text from the papers. 


\customHeader{2}{\bertroberta{}}
\label{06_bert_roberta}

This model was introduced in \mytextcite{roberta} and is an extension and improvement of the original \gls{bert} model. \bertroberta{} follows the same architecture as \gls{bert} but undergoes a more extensive pre-training process, specifically, it uses much larger batch sizes, longer sequences, and removes the Next Sentence Prediction" task used in \gls{bert}. Additionally, it is trained on 10 times more data than \gls{bert}, by merging five datasets: Google’s BooksCorpus, the English Wikipedia, Common-Crawl\footnote{\url{https://commoncrawl.org/}} English News \myparencite{ccnews_dataset}, OpenWebText \myparencite{openwebtext}, and the ``Stories" subset of the Common-Crawl \myparencite{stories_dataset}. It uses a \gls{bpe} tokenizer.

\customHeader{2}{\bertxlmroberta{}}
\label{06_bert_xlmroberta}

This model was introduced in \mytextcite{xlm_roberta} as the multilingual version of \bertroberta{}. For training, the authors constructed the 100-CC corpus for training \bertxlmroberta{}, which consists of monolingual data in 100 languages from the Common-Crawl, and used a SentencePiece tokenizer \myparencite{sentencepiece}, which extends the classical \gls{bpe} tokenization algorithm. 





\customHeader{2}{Excluded Models}
\label{06_bert_excluded_models}

Initially, we considered three other \gls{bert} models; however, after careful consideration, we decided not to include them due to various reasons.



\begin{paragraph}{ChouBERT}
    This model was introduced in
    \mytextcite{choubert} as a specialized model for Plant Health Monitoring. However, it was trained exclusively on French texts, which raises strong concerns about its ability to generalize to our linguistically and thematically diverse dataset.
\end{paragraph}

\begin{paragraph}{DistilBERT}
    This model was introduced in
    \mytextcite{distilbert} as a smaller version of the original \gls{bert} model. It uses a process called ``distillation", which compresses the knowledge of a larger \gls{bert} model into a smaller one, while still retaining decent performance. 

    During our preliminary training experiments with the \gls{bert} models mentioned above, DistilBERT consistently underperformed. Regardless of the configurations used, it defaulted to predicting the majority class. Therefore, we chose to omit DistilBERT from our final model selection.

\end{paragraph}


\begin{paragraph}{BioBERT}
    This model was introduced in \mytextcite{biobert} as a \gls{bert} variant specifically designed for biomedical text and tasks. Once again, this model uses the classical \gls{bert} architecture on PubMed abstracts in English. Since it has been vastly outperformed by \bertbiolinkbert{} in several \gls{nlp} tasks \myparencite{linkbert_biolinkbert}, we decided not to include it in our final model selection.
\end{paragraph}


% choubert
% distilbert
% biobert