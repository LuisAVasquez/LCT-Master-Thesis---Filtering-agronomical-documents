\customHeader{1}{Motivation}
\label{01_motivation}


\gls{nlp} is a subfield of \gls{ai} that focuses on the interaction between computers and humans through natural language. Its primary aim is to enable computers to understand, interpret, and generate human language in a way that is both meaningful and useful.
Recently, \gls{nlp} has been revolutionized by the widespread adoption of \neuralNetwork{}s. 
As in many domains, \gls{ai} and \gls{nlp} have been starting to be adopted in the Biomedical domain, with promising results \myparencite{BioMedicalBERT_ALBERT_ELECTRA_2021,BioNER_with_multilingualBERT_2019}. 

%Within \gls{nlp}, \textclassification{} is the process of categorizing textual data into predefined labels or categories based on its content. It's a common task in \gls{nlp} used for applications such as spam detection, sentiment analysis, and topic labeling \myparencite{survey_general_text_classificaiton}.


In this thesis, we present an application of \textclassification{} to Plant Health Surveillance, which refers to the systematic observation, detection, and analysis of plant diseases and pests to prevent their spread and ensure the health and productivity of plants. It involves tracking disease patterns, assessing risks, and implementing measures to protect plant ecosystems and agricultural systems.





\customHeader{1}{Hosting Institution : \INRAE{} Laboratory}
\label{01_2_1_INRAE}

\href{https://www.inrae.fr/}{\INRAE{} }\footnote{\url{https://www.inrae.fr/}}  (\emph{Institut National de Recherche} \emph{pour l'Agriculture, l'Alimentation et l'Environnement}) is a French research institute dedicated to agriculture, food, and the environment. 

\INRAE{} conducts scientific research and innovation activities to address various challenges related to sustainable agriculture, food production, and environmental conservation. Its research efforts span a wide range of disciplines, including agronomy, biology, ecology, genetics, forestry, hydrology, and applied mathematics and computer science.

Through its multidisciplinary approach and emphasis on sustainability, \INRAE{} plays a crucial role in driving scientific progress, innovation, and policy development in France and beyond, with the ultimate aim of ensuring sustainable and resilient agricultural and environmental systems.

Within \INRAE{}, the Mathematics and Numerics division (\href{https://www.inrae.fr/departements/mathnum}{MathNum}\footnote{\url{https://www.inrae.fr/departements/mathnum}}) focuses on advancing research in various fields including applied mathematics, statistics, bioinformatics, \gls{ai}, and information technology. The division's research units engage in theoretical, methodological, and applied research. They actively collaborate with teams from different divisions within \INRAE{} and external organizations, fostering interdisciplinary partnerships. \href{https://maiage.mathnum.inrae.fr}{\MAIAGE{}}\footnote{\url{https://maiage.mathnum.inrae.fr}}, one of these research units, is associated to the Paris-Saclay University, and brings together mathematicians, computer scientists, bioinformaticians, and biologists to address challenges in the fields of biology, agronomy, and ecology. This project was carried out with alongside the  \href{https://maiage.mathnum.inrae.fr/fr/bibliome}{\bibliome{}}\footnote{\url{https://maiage.mathnum.inrae.fr/fr/bibliome}} team, which specializes in the advancement and adaptation of \gls{nlp} and \gls{ml} techniques tailored for textual data within the fields of biology and agronomy.  

