\customHeader{1}{Overarching Project: TIERS-ESV}
\label{01_pesv_platform}
%\label{01_projects}



%\customHeader{2}{The \PESV{} Platform}
%\label{01_pesv_platform}
The \bibliome{} team, as part of the \gls{tiersesv} project\footnote{\url{https://plateforme-esv.fr/node/24638} and \url{https://maiage.inrae.fr/index.php/fr/node/2111}} (\emph{Processing of Health Risk Information and Knowledge for Epidemiological Surveillance in Plant Health}), collaborates frequently with the 
 \gls{pesv}\footnote{\url{https://plateforme-esv.fr/}} (\emph{Plant Health Epidemiological Surveillance Platform}),  and it is within this collaboration that we conducted this work.


The \gls{pesv} Platform was created in 2018 by 
7 French national public and private actors (\INRAE{},
the Anses Laboratory\footnote{\url{https://www.anses.fr/fr}}, 
the French Ministry of Agriculture, the Cirad Center\footnote{\url{https://www.cirad.fr/}}, the Acta reseach association \footnote{\url{https://www.acta.asso.fr/}},
the French Chamber of Agriculture \footnote{\url{https://chambres-agriculture.fr/}}, 
and the FREDON Network\footnote{\url{https://fredon.fr/}})  with  the objective to ensure the efficiency of epidemiological surveillance in plant health  \myparencite{convention_cadre_pesv}.


According to its mission statement, `the \gls{pesv} Platform leverages its scientific and technical expertise to engage in three key areas: monitoring, analysis, and advisory services. Its services cater to both public policies and all professionals within the plant health sector'.

As an integral part of its monitoring endeavors, the \gls{pesv} Platform includes the \gls{vsi}\footnote{\url{https://plateforme-esv.fr/thematiques/GTVSI}} (\emph{International Health Monitoring})  project. The main focus of the \gls{vsi} team is ``to address potential threats that may impact plant health by engaging in monitoring activities on a global scale". Its responsibilities encompass ongoing surveillance of events like notifications, reports, and changes in surveillance strategies across the globe. Furthermore, it conducts scientific monitoring on diverse subjects, including phylogeography, spatial distributions of species, habitat suitability assessments, innovative disease prevention measures, as well as surveillance and control methodologies.



\iffalse
\customHeader{2}{The \sanba{} Metaprogram}
\label{01_sanba_metaprogram}

\INRAE{} has research projects targeting strategic objectives, with a focus on fostering interdisciplinary and inter-divisional collaboration within the institute. These projects are referred to as ``metaprograms", one of which is the 
\gls{sanba}\footnote{\url{https://sanba.hub.inrae.fr/}} (\emph{Health and Welfare of Livestock Animals}) metaprogram. This project proposes a shift in perspective about livestock, transitioning from a vision where health and welfare were considered peripheral, to one that places the well-being of both animals at the hearth of designing more sustainable systems.

\todo{
Explain with which team from Sana we are working. 
}

\todo{What are their specific objectives?}
\fi