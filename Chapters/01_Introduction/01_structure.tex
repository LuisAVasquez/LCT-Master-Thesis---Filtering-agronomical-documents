\customHeader{1}{Structure of this Thesis}
\label{01_thesis_structure}

After this Introduction,  \headerName{} \ref{02_background} provides foundational knowledge and context for understanding our work. \headerName{} \ref{03_related_work} offers a review of existing literature, highlighting previous methodologies and their gaps. In \headerName{} \ref{vsi_dataset} we delve into the Dataset, detailing its sources and characteristics, and then continue to \headerName{} \ref{vsi_preprocessing}, which elucidates the preprocessing techniques employed. \headerName{} \ref{06_methodology} outlines the research methods and tools used, leading to \headerName{} \ref{07_results_and_discussion}, where our findings are presented and analyzed. \headerName{} \ref{08_practical_implications} sheds light on the real-world relevance of the findings, and the study culminates with \headerName{} \ref{09_conclusion}, summarizing key takeaways. % and potential future directions. 
Additionally, an Appendix is provided, rich with tables and detailed statistics, serving as a valuable resource for reference.

This thesis is submitted in fulfillment of the requirements for the Erasmus Mundus \href{https://lct-master.org/}{Language And Communication Technology}\footnote{\url{https://lct-master.org/}} from the European Union, as a student assigned to the University of Lorraine (France), and the University of the Basque Country (Spain).