\customHeader{1}{Preprocessing the \VSI{} dataset}
\label{vsi_preprocessing}

In response to the challenges outlined in \headerName{} \ref{vsi_data_issues}, we implemented multiple strategies that allowed us to obtain a preprocessed dataset suitable for \textclassification{}.

% %%%%%%%%%%%%%%%%%%%%%%%%%%%%%%%%%%%%%%%%%%%%%%%%%%%%%%%%%%%%%%%%%%%%%%%%%%%%%%%%%%%%%%
\customHeader{2}{Leveraging Keywords}
\label{vsi_leveraging_keywords}

Leveraging the available keywords and key phrases used for the Google Search queries to construct the dataset (Table \ref{tab:04_query_keywords}), we obtained four more sources of content. 


We opted to extract sentences containing keywords or key phrases from the content. Using the Python package \texttt{NLTK} \myparencite{nltk}, we divided each document into phrases and selected those containing at least one of the keywords or key phrases. This process was carried out for both the \trafilaturaAbstract{} and \trafilaturaFulltext{} sources of content. However, considering the diverse range of languages in the documents, it was not always apparent whether we should discard the content entirely or retain it. For example, even if the keyword is not present in the text, maybe its translation is; or, the language could use a non-Latin script. Consequently, we pursued two different approaches: in one case, we discarded the original content (O.C.), and in the other, we retained it, resulting in four new sources of content (Table \ref{tab:05_pesv_sources of content after preprocessing}).

\begin{table}%[]
    \centering
    \begin{tabular}{c|c}
       \textbf{\contentType{}}  & \textbf{Tool} \\ \hline
       \trafilaturaTitle{}  & Trafilatura \\
       \trafilaturaAbstract{}  & Trafilatura \\
       \trafilaturaFulltext{}  & Trafilatura \\
       \translationTitle{}  & Google Translate \\
     \keyphrasesAbstractOnly{} & Preprocessing \\
     \keyphrasesAbstractOC{} & Preprocessing \\
     \keyphrasesFulltextOnly{} & Preprocessing \\
     \keyphrasesFulltextOC{} & Preprocessing \\ 
    \end{tabular}
    \caption{Sources of text content in the \VSI{} Dataset after preprocessing}
    \label{tab:05_pesv_sources of content after preprocessing}
\end{table}


Figure \ref{fig:05_unique_entries_keyphrases_vsi} offers a comparison of the number of unique entries per content source. In both cases, keeping only the phrases containing keywords dramatically reduces the size of the data. Note that, in the case of keeping the original content, there are slightly less unique entries for both the \trafilaturaAbstract{} and the \trafilaturaFulltext{} ($19939-19798 = 141$ and $26346-24447=1899$, respectively). This may be explained by the fact that some data noise may be removed by the sentence extraction process, especially those cases where several entries differ by a small amount of tokens (see Table \ref{tab:04_website_names_as_suffixes}).

\begin{figure}
    \centering
    \includegraphics[width=0.750\textwidth]{Figures/05/Unique entries per content source_keyphrases.png}
    \caption{Unique entries per source of content in the \VSI{} dataset}
    \label{fig:05_unique_entries_keyphrases_vsi}
\end{figure}


% %%%%%%%%%%%%%%%%%%%%%%%%%%%%%%%%%%%%%%%%%%%%%%%%%%%%%%%%%%%%%%%%%%%%%%%%%%%%%%%%%%%%%%
\customHeader{2}{Data Cleaning}
\label{vsi_data_cleaning}



\customHeader{3}{Noise Removal}
\label{vsi_string_cleaning}

To address the text noise mentioned in \headerName{} \ref{vsi_issues_data_noise}, we use regular expressions to:

\begin{enumerate}
    \item Remove characters not in any human alphabet.
    \item Remove emojis, URLs, HTML tags, hours, dates, extra white spaces.
    \item Remove punctuation at beginning and end of the string.
    \item Remove digits at beginning and end of the string.
    \item Standardize quotation characters into the ASCII simple quotation character (').
    \item Standardize hyphen-like characters into the ASCII hyphen (-).
    \item Remove sentence suffixes that may be the name of the website, by deleting short text ($\leq 3$ tokens) that follow an ASCII hyphen or bar (|) near the end of the string.
\end{enumerate}

See Table \ref{appendix02:tab:pesv_string_cleaning_regex} in the \appendixname{} for some examples of the regular expressions used to clean the multilingual strings. Figure \ref{fig:05_multilingual_string_cleaning} shows the effect of string cleaning on an example text.

\begin{figure}
    \centering
    \includegraphics[width=\textwidth]{Figures/05/05_multilingual_string_cleaning.png}
    \caption{Removing noise from multilingual strings}
    \label{fig:05_multilingual_string_cleaning}
\end{figure}


\customHeader{3}{Deleting Error Messages}
\label{vsi_deleting_error_messages}

Fortunately, error messages in the dataset are very consistent. Given that they do not provide information on the relevance of a document, we may delete them using regular expressions. See Table \ref{appendix02:tab:error_messages} in the \appendixname{} for some examples of the regular expressions used to remove the error messages. 

\begin{table}%[]
    \centering
    \begin{tabular}{|lcc|}
    \hline
    \textbf{\contentType{}} & \textbf{Error Count} & \textbf{Percentage of Raw dataset} \\ \hline
    \trafilaturaTitle{} & 7224 & 20.89\% \\
    \trafilaturaAbstract{} & 12624 & 36.50\% \\
    \trafilaturaFulltext{} & 6609 & 19.11\% \\
    \translationTitle{} & 7232 & 20.91\% \\ \hline
    All \contentType{}s & 6399 & 18.50\% \\
    \hline
    \end{tabular}
    \caption{Error Message Count per \contentType{} in the \VSI{} dataset}
    \label{tab:05_vsi_error_message_count}
\end{table}


In Figure \ref{tab:05_vsi_error_message_count}, we present the number of error messages identified using our regular expressions. Notably, the proportion of error messages generally hovers around one-fifth of all entries. In $18.50\%\ (\approx 20\%)$ of cases \trafilatura{} completely fails to obtain any content from the website. This proportion almost doubles ($36.50\approx 40\%$) for the \trafilaturaAbstract{}. This discrepancy in the \trafilaturaAbstract{} could be attributed to the common occurrence of missing metadata, as discussed in \headerName{} \ref{vsi_data_statistics}.


\customHeader{3}{Handling scrapping errors}
% first attempt with title names regex. failed ~ 300 docs fewer
% second attempt: by length


In order to handle scraping errors from \trafilatura{}, such as obtaining the name of the website or certain HTML data headers, our first approach was using regular expressions to filter them out. After removing the error messages from all sources of content, we proceeded to count the occurrences of each entry and ranked them in descending order. Given that our dataset contains approximately 35,000 documents, we focused on entries that appeared at least twice.

Although we attempted to identify common patterns and create regular expressions to match frequent entries, this approach proved unsuccessful. It only filtered around 500 unique entries, accounting for a mere $2\%$ of all unique entries in the best case (\trafilaturaAbstract{}). Consequently, it became apparent that several scraping errors likely appeared only once in the dataset, rendering our previous strategy ineffective.

Nevertheless, along this examination, we noticed that scraping errors tend to be short in length (see Tables \ref{tab:04_headers} and \ref{tab:04_website_names}). As a result, we devised a new filtering strategy based on the length of the entries. However, simply counting naive tokens was insufficient, as languages like Chinese often lack whitespace in their text. Therefore, we needed a more robust algorithm to determine when a string is considered suspiciously ``too short."

Our algorithm to filter by length consists in:

\begin{enumerate}
    \item First, the string is split on white spaces.
    \item Then, if there is only one token, and the string has less than 20 characters, it is considered ``too short".
    \item Next, if there are less than four tokens, the string is considered ``too short"
    \item Finally, in all other cases, we keep the string.
\end{enumerate}


% %%%%%%%%%%%%%%%%%%%%%%%%%%%%%%%%%%%%%%%%%%%%%%%%%%%%%%%%%%%%%%%%%%%%%%%%%%%%%%%%%%%%%%
\customHeader{2}{Resolving Inconsistencies and Duplicate Documents}
\label{vsi_resolving_inconsistencies}

The initial preprocessing steps focused solely on cleaning the textual content of the dataset. Now, our attention turns to addressing the labeling challenges, specifically, instances where entries have identical content but differ in subject assignment or lack a subject (as in Tables \ref{tab:04_xmol_inconsistencies} and \ref{tab:04_annotation_inconsistencies}).

Considering our main goal is to automate the identification of articles relevant to epidemiologists, where subject assignment signifies their interest, we propose the following strategy:

\begin{tcolorbox}[colback=mylightblue,colframe=gray!50!black]
For each of the eight content sources, we group the dataset based on the text content. If a particular text has been assigned a subject at least once, we categorize it as \textbf{relevant}. Conversely, if no subject has been assigned to a text, we consider it \textbf{irrelevant}.
\end{tcolorbox}

We believe this strategy reflects the intention of the annotators as they labeled the documents (see \headerName{} \ref{vsi_data_annotation}). Additionally, by design, this strategy eliminates duplicate content. Table \ref{tab:determining_relevance} illustrates this process for the content from the \trafilaturaTitle{}.

\begin{table}
    \centering
    \resizebox{0.75\textwidth}{!}{
    \begin{tabular}{cp{10cm}c}
    \textbf{Entry ID} & \textbf{\trafilaturaTitle{}} & \textbf{Subject} \\ \hline
    4662 & Cousin of crop-killing bacteria mutating rapidly & None \\
    5885 & Cousin of crop-killing bacteria mutating rapidly & 4472 \\ \hline
    19661 & Modeling climate change impacts on potential  \ldots & None\\
    19361 & Modeling climate change impacts on potential \ldots - PubMed & None\\ \hline
    58 & Danger pour les végétaux : première détection de \ldots & None \\
    850 & Danger pour les végétaux : première détection de \ldots & 4259 \\ \hline
    26873 & Commodity risk assessment of ash logs from the US \ldots & 5524 \\
    27196 & Commodity risk assessment of ash logs from the US \ldots & 5530 \\ \hline
    29896 & Tornano le Giornate Fai di Primavera \ldots & None  \\
    29899 & Tornano le Giornate Fai di Primavera \ldots - LegnanoNews & None \\\hline
    \vdots & \vdots & \vdots

    \end{tabular} 
    }
    \vspace{0.25cm}
    \begin{center}
       \huge{$\Downarrow$}
    \end{center}
    \vspace{0.25cm}

    \centering
    \resizebox{0.75\textwidth}{!}{
    \begin{tabular}{p{10cm}c}
    
    \textbf{\trafilaturaTitle{}} & \textbf{Relevance} \\ \hline
    Cousin of crop-killing bacteria mutating rapidly & $1$ \\ \hline
    Modeling climate change impacts on potential  \ldots & $0$ \\ \hline
    Danger pour les végétaux : première détection de \ldots & $1$ \\ \hline
    Commodity risk assessment of ash logs from the US \ldots & $1$ \\\hline
    Tornano le Giornate Fai di Primavera \ldots & $0$ \\\hline
    \vdots & \vdots
    \end{tabular}
    }
    \caption{Determining Relevance for \trafilaturaTitle{}}
    \label{tab:determining_relevance}
\end{table}



% %%%%%%%%%%%%%%%%%%%%%%%%%%%%%%%%%%%%%%%%%%%%%%%%%%%%%%%%%%%%%%%%%%%%%%%%%%%%%%%%%%%%%%
\clearpage
\customHeader{2}{Preprocessing Results}
\label{vsi_results_of_preprocessing}

In this \headerName{}, we present the outcomes of our preprocessing techniques, yielding refined datasets suitable for text classification. Additional numerical results can be found in \appendixname{} \ref{appendix03:vsi_preprocessing results}.

We now proceed to present the results of our preprocessing methods, which produce clean datasets fit to be used for text classification.
Supplementary numerical results are available in the appendix.

\begin{figure}
    \centering
    \includegraphics[width=0.75\textwidth]{Figures/05/Preprocessing_Unique entries per content source.png}
    \caption{Unique Entries per content source before and after preprocessing}
    \label{fig:05_unique_entries_after_preprocessing}
\end{figure}

\begin{table}[ht]
\centering
\resizebox{0.75\textwidth}{!}{
\begin{tabular}{|l|c|}
\hline
\textbf{Unique entries per \contentType{}} & \textbf{Percentage left} \\
\hline
\trafilaturaTitle{}                                 & 67.49\%          \\
\trafilaturaAbstract{}                              & 85.13\%          \\
\trafilaturaFulltext{}                              & 85.53\%          \\
\translationTitle{}                                 & 68.65\%          \\
\hline
\keyphrasesAbstractOnly{}                           & 85.95\%          \\
\keyphrasesAbstractOC{}                             & 85.02\%          \\
\keyphrasesFulltextOnly{}                           & 87.17\%          \\
\keyphrasesFulltextOC{}                             & 85.41\%          \\
\hline
\end{tabular}
}
\caption{Unique Entries per content source before and after preprocessing}
\label{tab:05_unique_entries_after_preprocessing}
\end{table}




\begin{figure}[ht]
    \centering
    \subfigure[\trafilaturaTitle{}]{
        \begin{adjustbox}{max height=0.18\textheight}
            \includegraphics[width=0.45\textwidth]{Figures/05/Histograms/preprocessing length histogram parsed_trafilatura_title.png}
        \end{adjustbox}
        %\label{fig:image1}
    }
    \hfill
    \subfigure[\trafilaturaAbstract{}]{
            \begin{adjustbox}{max height=0.18\textheight}
            \includegraphics[width=0.45\textwidth]{Figures/05/Histograms/preprocessing length histogram parsed_trafilatura_abstract.png}
        \end{adjustbox}

        %\label{fig:image2}
    }
    
    \subfigure[\trafilaturaFulltext{}]{
        \begin{adjustbox}{max height=0.18\textheight}
          \includegraphics[width=0.45\textwidth]{Figures/05/Histograms/preprocessing length histogram parsed_trafilatura_fulltext.png}
        \end{adjustbox}
        %\label{fig:image3}
    }
    \hfill
    \subfigure[\translationTitle{}]{
            \begin{adjustbox}{max height=0.18\textheight}
                    \includegraphics[width=0.45\textwidth]{Figures/05/Histograms/preprocessing length histogram translation_title.png}
        \end{adjustbox}
        %\label{fig:image4}
    }
    
    
    %%
    \subfigure[\keyphrasesAbstractOnly{}]{
        \begin{adjustbox}{max height=0.18\textheight}
            \includegraphics[width=0.45\textwidth]{Figures/05/Histograms/preprocessing length histogram sentence_with_keywords_parsed_trafilatura_abstract_only_relevant_sentences.png}
        \end{adjustbox}
        
        %\label{fig:image2}
    }
    \hfill
    \subfigure[\keyphrasesAbstractOC{}]{
         \begin{adjustbox}{max height=0.18\textheight}
         \includegraphics[width=0.45\textwidth]{Figures/05/Histograms/preprocessing length histogram sentence_with_keywords_parsed_trafilatura_abstract_keep_original_content.png}
        \end{adjustbox}
    
        %\label{fig:image1}
    }

    
    \subfigure[\keyphrasesFulltextOnly{}]{

    \begin{adjustbox}{max height=0.18\textheight}
           \includegraphics[width=0.45\textwidth]{Figures/05/Histograms/preprocessing length histogram sentence_with_keywords_parsed_trafilatura_fulltext_only_relevant_sentences.png}
        \end{adjustbox}
    
        %\label{fig:image4}
    }
    \hfill
        \subfigure[\keyphrasesFulltextOC{}]{
        \begin{adjustbox}{max height=0.18\textheight}
            \includegraphics[width=0.45\textwidth]{Figures/05/Histograms/preprocessing length histogram sentence_with_keywords_parsed_trafilatura_fulltext_keep_original_content.png}
        \end{adjustbox}
        
        %\label{fig:image3}
    }

    
    \caption{Token distribution after preprocessing the \VSI{} dataset}
    \label{fig:05_vsi_token_distribution_after_preprocessing}
\end{figure}





\begin{table}[ht]
\centering
\resizebox{\textwidth}{!}{
\begin{tabular}{|l|c|c|}
\hline
\textbf{\contentType{}} & \textbf{Before preprocessing} & \textbf{After preprocessing} \\
\hline
\trafilaturaTitle{} & 11.29  & 12.46 \\
\trafilaturaAbstract{}  & 50.02     & 50.55     \\
\trafilaturaFulltext{}    & \textcolor{orange}{1102.29}     & \textcolor{orange}{1097.76}    \\
\translationTitle{}   & 11.66     & 12.82     \\ \hline
\keyphrasesAbstractOnly{}  & (50.02)     & 34.92     \\
\keyphrasesAbstractOC{} & (50.02)     & 35.77     \\

\keyphrasesFulltextOnly{} & (1102.29)     & 115.66    \\
\keyphrasesFulltextOC{} & (1102.29)     & 217.71    \\

\hline
\end{tabular}
}
\caption{Mean token counts before and after preprocessing the \VSI{} dataset}
\label{tab:05_vsi_mean_tokens_before_after_preprocessing}
\end{table}





\begin{figure}[ht]
    \centering
    \subfigure[\trafilaturaTitle{}]{
        \begin{adjustbox}{max height=0.18\textheight}
            \includegraphics[width=0.45\textwidth]{Figures/05/Charts/balance Title.png}
        \end{adjustbox}
        
        %\label{fig:image1}
    }
    \hfill
    \subfigure[\trafilaturaAbstract{}]{
        \begin{adjustbox}{max height=0.18\textheight}
            \includegraphics[width=0.45\textwidth]{Figures/05/Charts/balance Abstract.png}
        \end{adjustbox}
        
        %\label{fig:image2}
    }
    
    \subfigure[\trafilaturaFulltext{}]{
        \begin{adjustbox}{max height=0.18\textheight}
            \includegraphics[width=0.45\textwidth]{Figures/05/Charts/balance Fulltext.png}
        \end{adjustbox}
        
        %\label{fig:image3}
    }
    \hfill
    \subfigure[\translationTitle{}]{
        \begin{adjustbox}{max height=0.18\textheight}
            \includegraphics[width=0.45\textwidth]{Figures/05/Charts/balance Translated Title.png}
        \end{adjustbox}
        %\label{fig:image4}
    }
    
    
    %%
    \subfigure[\keyphrasesAbstractOnly{}]{
        \begin{adjustbox}{max height=0.18\textheight}
            \includegraphics[width=0.45\textwidth]{Figures/05/Charts/balance Sentence with Keywords Abstract.png}
        \end{adjustbox}
        %\label{fig:image2}
    }
    \hfill
    \subfigure[\keyphrasesAbstractOC{}]{
        \begin{adjustbox}{max height=0.18\textheight}
            \includegraphics[width=0.45\textwidth]{Figures/05/Charts/balance Sentece with Keywords and OC Abstract.png}
        \end{adjustbox}
        %\label{fig:image1}
    }

    
    \subfigure[\keyphrasesFulltextOnly{}]{
        \begin{adjustbox}{max height=0.18\textheight}
            \includegraphics[width=0.45\textwidth]{Figures/05/Charts/balance Sentence with keywords Fulltext.png}
        \end{adjustbox}
        %\label{fig:image4}
    }
    \hfill
    \subfigure[\keyphrasesFulltextOC{}]{
        \begin{adjustbox}{max height=0.18\textheight}
            \includegraphics[width=0.45\textwidth]{Figures/05/Charts/balance Sentence with keywords and OC Fulltext.png}
        \end{adjustbox}
        %\label{fig:image3}
    }

    
    \caption{Positive/Negative balance after preprocessing the \VSI{} dataset}
    \label{fig:05_vsi_pos_neg_balance}
\end{figure}




% making sure to print all tables and figures 
\clearpage