En el \'area de la Vigilancia Epidemiol\'ogica de la Salud de las Plantas, el an\'alisis preciso de reportes escritos sobre eventos que afectan a la agricultura es crucial. Este Trabajo de Fin de M\'aster en An\'alisis y Procesamiento del Lenguaje aprovecha las capacidades de la Clasificaci\'on Autom\'atica de Textos, espec\'ificamente, enfoc\'andose en su aplicaci\'on al monitoreo la salud de las plantas. Realizado en el marco del proyecto \emph{TIERS-ESV}, este trabajo es el resultado de una colaboración entre el equipo Bibliome de la unidad MaIAGE del laboratorio INRAE y el equipo VSI de la Plataforma de Vigilancia de la Salud de las Plantas PESV. Nuestra investigaci\'on se apoya en un conjunto de datos curados y anotados por el equipo VSI, que consiste en reportes extra\'idos de fuentes recopiladas en l\'inea. Este conjunto diverso y multiling\"ue de datos fue sometido a un preprocesamiento exhaustivo, aplicando m\'etodos para la eliminación de ruido, la supresi\'on de mensajes de error y el tratamiento de errores de recopilaci\'on y discrepancias en las anotaciones. Utilizamos diversos modelos BERT para la Clasificación de Textos, adaptados a nuestro conjunto de datos mediante \finetuning{} y un m\'etodo de entrenamiento basado en plantillas. Despu\'es de un entrenamiento intensivo de los clasificadores, seleccionamos los modelos de mejor rendimiento. La eficacia de los modelos obtenidos como resultado de este trabajo conducir\'a a la implementaci\'on de clasificadores basados en BERT, preparados para asistir a los expertos del equipo VSI en su misi\'on de monitoreo.