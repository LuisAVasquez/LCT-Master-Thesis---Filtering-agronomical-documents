\\[20pt]

Dans le domaine de la Surveillance \'Epid\'emiologique de la Sant\'e des Plantes, l'analyse pr\'ecise des rapports \'ecrits sur les \'ev\'enements affectant l'agriculture est cruciale. 
Ce m\'emoire de ma\^itrise en Traitement Automatique des Langues exploite les capacit\'es de Classification Automatique des Textes, en se concentrant sur son application dans la surveillance de la sant\'e des plantes. R\'ealis\'e dans le cadre du projet \emph{TIERS-ESV} ce travail est le fruit d'une collaboration entre l'\'equipe Bibliome de l'unit\'e INRAE MaIAGE et l'\'equipe VSI de la Plateforme de Surveillance de la Sant\'e des Plantes PESV. Notre recherche s'appuie sur un ensemble de donn\'ees cur\'ees et annot\'ees par l'\'equipe VSI, qui consiste en des rapports extraits de sources collect\'ees en ligne. Cet ensemble de donn\'ees diversifi\'e et multilingue a \'et\'e soumis \`a un pr\'etraitement approfondi, en appliquant des m\'ethodes d'\'elimination du bruit, de suppression des messages d'erreur et de traitement des erreurs de mise au rebut et des divergences d'annotation. Nous avons utilis\'e divers mod\`eles BERT pour la Classification des Textes, adapt\'es \`a notre ensemble de donn\'ees par le biais d'un finetuning et l'Apprentissage par Exploitation de Motifs \myparencite{pet_paper}. 
Apr\`es un entra\^inement intensif \`a la classification, nous avons s\'electionn\'e les mod\`eles les plus performants. 
L'efficacit\'e des mod\`eles cr\'e\'es dans le cadre de ce travail conduira \`a l'utilisation de classificateurs bas\'es sur \BERT{} pour aider les experts de l'\'equipe VSI dans leur tâche de surveillance.
