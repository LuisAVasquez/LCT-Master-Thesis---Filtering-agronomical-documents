
\begin{table}[!htbp]
  \centering


  \begin{tabular}{|c|p{0.6\linewidth}|c|}
    \hline
    \textbf{Entry ID} & \textbf{\trafilaturaTitle{}} & \textbf{Subject} \\
    \hline
    4662 & Cousin of crop-killing bacteria mutating rapidly & None \\
    5885 & Cousin of crop-killing bacteria mutating rapidly & 4472 \\ \hline
    %1939 & APHIS Updates Federal Domestic Soil Quarantine Map & None \\
    %6966 & APHIS Updates Federal Domestic Soil Quarantine Map & 4553 \\
    58 & Danger pour les végétaux : première détection de la bactérie Xylella fastidiosa dans le Gard & None \\
    850 & Danger pour les végétaux : première détection de la bactérie Xylella fastidiosa dans le Gard & 4259 \\ \hline
    26873 & Commodity risk assessment of ash logs from the US treated with sulfuryl fluoride to prevent the entry of the emerald ash borer Agrilus planipennis & 5524 \\
    27196 & Commodity risk assessment of ash logs from the US treated with sulfuryl fluoride to prevent the entry of the emerald ash borer Agrilus planipennis & 5530 \\ \hline
    %17580 & An eco-epidemiological model supporting rational disease management of Xylella fastidiosa. An application to the outbreak in Apulia (Italy) & 5017 \\
    %17844 & An eco-epidemiological model supporting rational disease management of Xylella fastidiosa. An application to the outbreak in Apulia (Italy) & 5056 \\
    %2777 & An odorant binding protein mediates Bactrocera dorsalis olfactory sensitivity to host plant volatiles and male attractant compounds & 4426 \\
    %4984 & An odorant binding protein mediates Bactrocera dorsalis olfactory sensitivity to host plant volatiles and male attractant compounds & \\
    322 & Anche a Varese l'invasione della Popillia Japonica, l'insetto devastatore di campi e giardini & None \\
    335 & Anche a Varese l'invasione della Popillia Japonica, l'insetto devastatore di campi e giardini & 4286 \\
    343 & Anche a Varese l'invasione della Popillia Japonica, l'insetto devastatore di campi e giardini & 4286 \\
    %13369 & Breakthrough in protecting bananas from Panama disease & 4766 \\
    %13464 & Breakthrough in protecting bananas from Panama disease & 4806 \\
    %315 & Coleottero Popillia japonica: attivato il piano di controllo ERSAF & 4286 \\
    %1175 & Coleottero Popillia japonica: attivato il piano di controllo ERSAF & None \\
    \hline
  \end{tabular}

\vspace{10pt}

  \begin{tabular}{|c|p{0.6\linewidth}|c|}
    \hline
    \textbf{Entry ID} & \textbf{\trafilaturaAbstract{}}  & \textbf{Subject} \\
    \hline
    13997 & A CABI-led study involving 57 scientists from 46 different institutions \ldots%has provided a comprehensive review of the devastating fall armyworm (Spodoptera frugiperda) including details on its invasiveness, biology, ecology and management. 
    & 4829 \\
    24902 & A CABI-led study involving 57 scientists from 46 different institutions \ldots % has provided a comprehensive review of the devastating fall armyworm (Spodoptera frugiperda) including details on its invasiveness, biology, ecology and management. 
    & None \\ \hline
    4662 & A bacterial species closely related to deadly citrus greening disease \ldots % is rapidly evolving its ability to infect insect hosts, and possibly plants as well.
    & 4472 \\
    5388 & A bacterial species closely related to deadly citrus greening disease \ldots %is rapidly evolving its ability to infect insect hosts, and possibly plants as well. 
    & None\\ \hline
    22658 & Ministry of Agriculture activated batteries in the fight against xylella.\ldots % This is evident, at least, from recent data suggesting that bacteria-fighting work in the state of Alicante has accelerated since the European Union (EU) slap on the wrist for accumulated delays. And this, 94,000 almond trees destroyed in one year,… 
    & 5245 \\
    24067 & Ministry of Agriculture activated batteries in the fight against xylella.\ldots % This is evident, at least, from recent data suggesting that bacteria-fighting work in the state of Alicante has accelerated since the European Union (EU) slap on the wrist for accumulated delays. And this, 94,000 almond trees destroyed in one year,… 
    & 5331 \\ 
    \hline
  \end{tabular}

\vspace{10pt}

  \begin{tabular}{|c|p{0.6\linewidth}|c|}
    \hline
    \textbf{Entry ID} & \textbf{\trafilaturaFulltext{}} & \textbf{Subject} \\
    \hline
    32437 & Altre 23 piante infette dal batterio Xylella fastidiosa subsp. pauca ceppo ST53 sono state individuate tra Fasano (Brindisi) e Castellana Grotte (Bari), \ldots % grazie al monitoraggio disposto dall’Osservatorio fitosanitario regionale e condotto dall’Agenzia regionale per le attività irrigue e forestali della Regione Puglia (Arif). 23 piante positive rilevate dal monitoraggio Xylella. Come informa Infoxylella, sono stati appena pubblicati sul portale istituzionale della Regione Puglia, Emergenza Xylella, tre nuovi rapporti di prova che riportano la presenza di altre 23 piante positive. Di queste 15 olivi ricadono nel territorio di Fasano e appartengono a un grande focolaio, già noto, nei pressi di una stazione di servizio. Le altre otto piante (sette olivi e un mandorlo) sono state individuate nel territorio di Castellana Grotte, centro agricolo in cui erano già stati trovati alcuni olivi infetti. Con l’attuale monitoraggio aumentato numero positivi. Dai dati del cruscotto del sito Emergenza Xylella, aggiornato costantemente durante l’intera durata del monitoraggio, partito a giugno 2022, risulta che la superficie attualmente ispezionata è il 79,18\% di quella prevista e che le piante controllate sono invece il 77,76\% di quelle da campionare. Finora sono state piante analizzate 221.887 piante. Di queste 294 sono risultate infette e sono state già abbattute per oltre il 90\%. È interessante, infine, notare che il numero dei positivi dell’attuale monitoraggio risulta già superiore al doppio dei positivi intercettati nel precedente monitoraggio. Obbligo lavorazioni terreni nei Comuni sotto 200 m s.l.m. Intanto la Regione Puglia ha emanato la Circolare n. 1 del 28 marzo riguardante le misure fitosanitarie obbligatorie per ridurre la diffusione di Xylella e precisamente le lavorazioni dei terreni. Essa è rivolta ai proprietari/conduttori di terreni agricoli e ai proprietari/gestori di superfici agricole non coltivate nei Comuni con altitudine inferiore ai 200 metri sul livello del mare, dove l’insetto vettore del batterio Xylella, la sputacchina media, è prossima al raggiungimento del quarto stadio giovanile, e ricorda l’obbligo di eseguire le lavorazioni dei terreni il prima possibile e comunque non oltre il 24 aprile. Monitoraggio Xylella: altre 23 piante infette - Ultima modifica: 2023-03-31T09:33:57+02:00 da 
    & 5781 \\
    32712 & Altre 23 piante infette dal batterio Xylella fastidiosa subsp. pauca ceppo ST53 sono state individuate tra Fasano (Brindisi) e Castellana Grotte (Bari),\ldots %grazie al monitoraggio disposto dall’Osservatorio fitosanitario regionale e condotto dall’Agenzia regionale per le attività irrigue e forestali della Regione Puglia (Arif). 23 piante positive rilevate dal monitoraggio Xylella. Come informa Infoxylella, sono stati appena pubblicati sul portale istituzionale della Regione Puglia, Emergenza Xylella, tre nuovi rapporti di prova che riportano la presenza di altre 23 piante positive. Di queste 15 olivi ricadono nel territorio di Fasano e appartengono a un grande focolaio, già noto, nei pressi di una stazione di servizio. Le altre otto piante (sette olivi e un mandorlo) sono state individuate nel territorio di Castellana Grotte, centro agricolo in cui erano già stati trovati alcuni olivi infetti. Con l’attuale monitoraggio aumentato numero positivi. Dai dati del cruscotto del sito Emergenza Xylella, aggiornato costantemente durante l’intera durata del monitoraggio, partito a giugno 2022, risulta che la superficie attualmente ispezionata è il 79,18\% di quella prevista e che le piante controllate sono invece il 77,76\% di quelle da campionare. Finora sono state piante analizzate 221.887 piante. Di queste 294 sono risultate infette e sono state già abbattute per oltre il 90\%. È interessante, infine, notare che il numero dei positivi dell’attuale monitoraggio risulta già superiore al doppio dei positivi intercettati nel precedente monitoraggio. Obbligo lavorazioni terreni nei Comuni sotto 200 m s.l.m. Intanto la Regione Puglia ha emanato la Circolare n. 1 del 28 marzo riguardante le misure fitosanitarie obbligatorie per ridurre la diffusione di Xylella e precisamente le lavorazioni dei terreni. Essa è rivolta ai proprietari/conduttori di terreni agricoli e ai proprietari/gestori di superfici agricole non coltivate nei Comuni con altitudine inferiore ai 200 metri sul livello del mare, dove l’insetto vettore del batterio Xylella, la sputacchina media, è prossima al raggiungimento del quarto stadio giovanile, e ricorda l’obbligo di eseguire le lavorazioni dei terreni il prima possibile e comunque non oltre il 24 aprile. Monitoraggio Xylella: altre 23 piante infette - Ultima modifica: 2023-03-31T09:33:57+02:00 da 
    & 5820 \\
    \hline
  \end{tabular}
  \caption{Annotation inconsistencies in the \VSI{} dataset}
  \label{tab:04_annotation_inconsistencies}
\end{table}

